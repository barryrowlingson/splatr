\documentclass{article}

\title{{\sc Plasta}/{\sc Splatr} \\ Space-time models with partial likelihood}

\begin{document}

\maketitle

\section*{Naming}
Previous package developments in this area had names that reflected only the infectious modelling 
applications - namely {\sc Osiris} (Our SIR in Space-time)  and {\sc Plim} (Partial Likelihood Infectious Modelling). 
With new applications comes a new name - either {\sc Plasta} - Partial Likelihood Analysis for Space-Time Applications or
{\sc Splatr} - Space-time Partial Likelihood Analysis Tools for R.

\section*{Existing Package Connections}
The following packages may be of use or interest:
\begin{itemize}
\item {\tt surveillance} - contains some space-time modelling and data handling functionality.
\item {\tt sp} - for general spatial data structures.
\item {\tt spatstat} - for spatial point-patterns.
\end{itemize}

\section*{Outline}

We want to develop a package for modelling space-time point patterns, as in the FMD outbreak model,
the nesting birds model, or the swine-flu model. These three examples represent different aspects of the
same essential underlying process:

\begin{enumerate}
\item FMD: cases occur a maximum of once at a subset of possible points (such as farm locations).
\item Swine Flu: small numbers of cases (relative to the population) occur 
with a related spatial label (such as an electoral ward).
\item Bird Nest: cases occur anywhere on an island.
\end{enumerate}

The partial likelihood probability for each case, $i$, occurring at time $t_i$ of
these models is $\lambda_i/D(t_i)$, where $D(.)$ (for `Denominator') is:

\begin{enumerate}
\item for FMD, the sum of $\lambda_j(t_i)$ of all farms, $j$, that are susceptible at time $t_i$.
\item for swine flu, $\lambda_j(t_i)$ summed over all regions.
\item for bird nesting, the integral of $\lambda(t_i)$ over the nesting region.
\end{enumerate}

In each example there may be up two times of importance to each case as the case goes from susceptible
to infectious to removed:

\begin{enumerate}
\item for FMD, farms are infectious from a fixed number of days
  before report up to the culling date. All infected farms are
  culled. The dataset may also include farms that are culled without
  being reported infected (pre-emptive culling) and farms that are
  neither reported infected nor culled. Missing time events are coded as {\tt NA} values.
\item for swine flu, cases are deemed infectious for a fixed number of days.
\item for bird nesting, an arriving bird doesn't leave. Hence there is only one time point per event.
\end{enumerate}

Covariates for these examples are as follows:

\begin{enumerate}
\item for FMD, we have farm-level covariates for all farms in the study region
\item for swine flu we have region-level covariates for all regions
\item for bird nesting we have spatial covariates over the island
\end{enumerate}

For FMD and swine flu we break $\lambda$ into infectious, susceptible, and transmission components for
each infectious case:
$$
\lambda_{ij}(t) = A_i(t)B_j(t)f_{ij}(t)
$$

Covariate parameters for infectors appears in $A_i$ and that for susceptibles in $B_j$.
Typically $f_{ij}(t)$ is a function of some distance metric between farms or regions. This could be 
a simple Euclidean distance or some measure of travel time or association.

For bird nesting, $\lambda$ is a continuous function over the space which depends on the 
covariates and the locations of existing nests. The parameterisation of the dependence 
on the covariates and locations should be flexible.


\section*{FMD Examples}

The SMMR paper explores a number of possibilities for modelling the epidemic. For the infectious
and susceptible components the most general form is:
$$
A_i = (n_{0i}^\gamma + \alpha_1 n_{1i}^\gamma + \alpha_2 n_{2i}^\gamma + \dots) \exp(z'_i\delta)
$$

where $n_{ki}$ is the number of animals of type $k$ in farm $i$ and
the $\alpha$ parameters give the effect of those animals relative to
type $0$ animals. The $\exp(z'_i\delta)$ term allows for more general
farm covariates, either continuous such as farm area or altitude, or
categorical such as hill farm/lowland farm/urban smallholding.

For the transmission function, the following form for $f(u)$ is used:
$$
f(u) =  \exp(-(u/\phi)^\kappa) + \rho
$$

with $\kappa$ either variable or fixed at $0.5$ for partially pragmatic reasons.


\subsection*{Proposed Functions}

Given a data set with its first few rows looking like this:

\begin{verbatim}
           x      y  area ncattle npigs nsheep report   cull 
1855  350.85 557.59  15.5      25     0      0  28.08  34.87
2875  341.73 572.09 205.0     270     0    597     NA  34.71
4379  350.41 527.04  88.1     300     0    312  28.31  31.73 
2779  337.32 569.50  88.0     210     0      0  28.60  35.41 
2409  347.80 541.85 105.6     137     0    436     NA     NA
2485  356.90 541.12 137.5      69     0   1573  29.62  34.64 
\end{verbatim}

 we would like to be able to do:

\begin{verbatim}
> fit0 = fitSIR2(
           data, 
           times = ~I(report-5)+cull,
           distF = expDecay(~x+y,kappa=0.5,starting(phi=.2,rho=.01))
           )
\end{verbatim}

to fit a model with no covariate information. We can then add animal abundance covariates and
farm level covariates as appropriate:

\begin{verbatim}
> fit1 = fitSIR2(
           data,
           times=~I(report-5)+cull,
           cov=list(
              powerSum(~nsheep+ncattle,starting(1),gamma=1),
              logLinear(~area,starting(1))
              ),
           distF = expDecay(~x+y,kappa=0.5,starting(phi=.2,rho=.01)),
           ...
           )
\end{verbatim}

with the arguments as follows:

\begin{itemize}
\item {\tt data} the data frame as above
\item {\tt times} a formula for infection and removal times. Note the use of {\tt I(report-5)} to implement a 
five-day reporting delay
\item {\tt cov} a list of covariate expressions
\item {\tt distF} an expression for the distance-dependence function.
\item {\tt ...} further arguments to control the optimisation algorithm.
\end{itemize}

For more than two types of animal, and fitting the kappa value, this simply becomes:

\begin{verbatim}
> fit2 = fitSIR2(
           data,
           times=~I(report-5)+cull,
           cov=list(
              powerSum(~nsheep+ncattle+npigs,gamma=1,starting(1,1)),
              logLinear(~area,starting(1))
              ),
           distF = expDecay(~x+y,kappa=0.5,starting(phi=.2,rho=.01)),
           ...
           )
\end{verbatim}

If the $\kappa$ and $\gamma$ values are to be optimised over, this becomes:

\begin{verbatim}
> fit3 = fitSIR2(
           data,
           times=~I(report-5)+cull,
           cov=list(
              powerSum(~nsheep+ncattle+npigs,starting(1,1,gamma=1))
              logLinear(~area,starting(1))
              ),
           distF = expDecay(~x+y,starting(phi=.2,rho=.01,kappa=0.5)),
           ...
           )
\end{verbatim}

It should also be possible for the user to get to the underlying log-likelihood function for the data. 
We supply a function called \verb|llikSIR2| which returns the log-likelihood function:

\begin{verbatim}
> L0 = likSIR2(
           data, times = ~I(report-5)+cull,
           cov=list(
              powerSum(~nsheep+ncattle,gamma=1),
              logLinear(~area)
              ),
           distF = expDecay(~x+y,kappa=0.5)
           )
\end{verbatim}

Now \verb|L0| is a function, and its parameters are just the parameters of the model: $\alpha$, $\beta$, $\delta$, $\phi$, and $\rho$ -- 
and it returns the partial log-likelihood:

\begin{verbatim}
> L0(c(1,2,0.1,0.001,-2))
[1] -4534.02
\end{verbatim}

The user can then use \verb|L0| wherever they need to compute a log-likelihood, such as in likelihood profiles. They can also
plug this into \verb|optim| with only the \verb|optim| parameters to worry about (most importantly 
to set \verb|fnscale| to $-1$ so that \verb|optim| maximises the log-likelihood):

\begin{verbatim}
> optim(L0, c(1,2,0.1,.001,-2), method="BFGS",
        control=list(fnscale=-1,reltol=1e-3),hessian=TRUE)
\end{verbatim}

This gives the advanced user total freedom of control parameters and the option to
use newer algorithms from other R packages.


\subsection*{Support Functions}

Useful things to do with the data include:
\begin{itemize}
\item summary of the SIR state of the data at a given time point.
\item plot of the time-behaviour of the data showing numbers of points in each of the SIR states
\item map of the points at a given time point showing which points are in which SIR state
\item animation of the epidemic in progress
\item simulations
\end{itemize}

\subsection*{Extensions}

The model could be extended to cover the situation where farms can get multiple infections. The data would 
probably have to be supplied in `long' format with one row per infection, and another data frame with one row per farm 
with per-farm covariates. The model could possibly include the effect on infectious/susceptibility rates of farms that
have already had previous infections.


\section*{Swine Flu Examples}

For the Swine Flu example we have a data frame of cases with time and
region code, and a data frame of regions with covariates. The
covariates must include coordinates and the population of the
region. The infectious period is a fixed number of days starting from
the report date.

\subsection*{Proposed Functions}

Given a data frame of cases:

\begin{verbatim}
> head(caseData)
     region       Date Age sex
1 Zone-0015 2010-01-26  15   F
2 Zone-0015 2010-02-07  28   M
3 Zone-0016 2009-12-05  27   F
4 Zone-0016 2010-02-23  45   M
5 Zone-0016 2010-03-03  40   F
6 Zone-0016 2010-01-20  40   F
\end{verbatim}

and a data frame of region information:

\begin{verbatim}
> head(regionData)
          Population IMDScore fWhite    XC    YC
Zone-0015       4452       28   13.5 26400 17800
Zone-0016      10691       76   74.7 28000 17500
Zone-0017       2157       64   62.6 24500 15100
Zone-0020       7025       12   93.0 24900 13500
Zone-0021       8963       84   75.8 28800 14500
Zone-0022       3621       28    7.1 25600 17700
\end{verbatim}

To fit a purely distance-decay model we do:

\begin{verbatim}
fit1 = fitSIR1(
      start,
      cases = caseData, ~region+Date,
      locations = regionData, 
      distDecay = dDecay(~XC+YC),
      covar = ~1, pop = ~Population,
      tInfect = 5
      )
\end{verbatim}

To fit a purely covariate model we do:

\begin{verbatim}
fit2 = fitSIR1(
      start,
      cases = caseData, ~region+Date,
      locations = regionData, 
      distDecay = flat(1)
      covar = ~IMDScore+fWhite, pop = ~Population,
      tInfect = 5
      )
\end{verbatim}

To fit a full model with spatial distance dependence and covariates:

\begin{verbatim}
Ffull = fitSIR1(
      start,
      cases = caseData, ~region+Date,
      locations = regionData, 
      distDecay = dDecay(~XC+YC)
      covar = ~IMDScore+fWhite, pop = ~Population,
      tInfect = 5
      )
\end{verbatim}

\subsection*{Support Functions}

The following operations are useful:
\begin{itemize}
\item time-series plots of case numbers
\item time-series plots of cases in regions
\item maps of total cases
\item maps of cases at time points
\end{itemize}

\subsection*{Extensions}

Variations in the distance-decay model can be incorporated. For example to model
according to some matrices \verb|M(regions)| and \verb|N(regions)| as a linear expression we could do:

\begin{verbatim}
Fnet = fitSIR1(
            cases = caseData, ~region+Date,
            locations = regionData, 
            distDecay = dLinear(M(locations),N(locations)),
            covar = ~IMDScore+fWhite, pop = ~Population,
            tInfect = 5
            )
\end{verbatim}

This would allow the proposed extension using commuting number matrices or school journey
matrices to be included.



\section*{Bird Nesting Examples}

For the bird nesting example, the partial likelihood depends on the nature of the underlying spatial surface 
and the locations of existing bird nests. The paper experiments with different dependencies on the 
ground elevation and the functional dependence on nest distance. There are no covariates for the nests 
themselves.

\subsection*{Proposed Functions}

For these applications we propose a basic function called \verb|fitST|, for partial-likelihood space-time
models.

Given a data set such as:

\begin{verbatim}
> nests[1:4,]
           x     y arrive species
Nest-1 13490 23110     30  Puffin
Nest-2 13440 23730     31    Tern
Nest-3 13090 23720     38    Gull
Nest-4 12810 23370     44  Puffin
\end{verbatim}

and some function that returns the height at a given location:

\begin{verbatim}
> height(13440,23730)
[1] 21.7
\end{verbatim}

We should be able to do:

\begin{verbatim}
> fit1 = fitST(
   start=c(beta=c(0.1,-0.1), theta=40, phi=2),
   nests, times=~arrive, location=~x+y,
   spatialFX=~height+I(height^2), 
   spatialFunctions=list(height=height),
   eventFX = dDecay(neighs=1, c=1, d0=0.24),
   area=island,
   ...
   )
\end{verbatim}

Suppose we also have a function of location that returns a categorical variable describing surface nature at given points.
Here we show the nature of the ground at the nest sites:

\begin{verbatim}
> surface(nests$x,nests$y)
 [1] meadow rock   meadow grass  grass  meadow grass
 [8] grass  grass  meadow
Levels: grass meadow rock
\end{verbatim}

This can be incorporated as a spatial effect covariate, with two extra
beta parameters for the three-level factor:

\begin{verbatim}
> fit2 = fitST(
   start=c(beta=c(0.1,-0.1,.2,.3), theta=40, phi=2),
   nests, times=~arrive, location=~x+y,
   spatialFX=~height+I(height^2)+surface, 
   spatialFunctions=list(height=height,surface=surface),
   eventFX = dDecay(neighs=1, c=1, d0=0.24),
   area=island
   )
\end{verbatim}

with the factor parameters being relative to the first level of the factor (in this case, `grass').

As with FMD, we expose a way of creating a log-likelihood function:

\begin{verbatim}
> F = llikST(
   nests,
   times=~arrive+depart, location=~x+y,
   spatialFX=~height+I(height^2), 
   spatialFunctions=list(height=height),
   eventFX = dDecay(neighs=1, c=1, d0=0.24),
   area=island
   )
\end{verbatim}

This returns a function that computes the partial log-likelihood, in this case taking four parameters:
\begin{verbatim}
> F(c(beta=c(0.022,-0.046), theta=38.79, phi=2.82))
[1] -902.66
\end{verbatim}

\subsection*{Support Functions}

Useful things to do with the data include:
\begin{itemize}
\item summary of the state of the process at some time point
\item the intensity surface at some time point over a given space
\item map of the process at some time point, possibly superimposed on the intensity surface
\end{itemize}

\subsection*{Extensions}

If the species of nesting bird varies then a likelihood for a marked point pattern could be computed. This could
model whether birds of the same species nest together or whether certain species avoid each other. It might possibly 
also show whether birds of a species have an affinity for areas with particular covariate values (high areas, or rocky 
areas).


The function could also be extended to cope with situations where the birds nest for a time and then
leave by adding an extra component to the \verb|times| formula:

\begin{verbatim}
> fit3 = fitST(
   start,
   nests, times=~arrive+depart, location=~x+y,
   spatialFX=~height+I(height^2)+surface, 
   spatialFunctions=list(height=height,surface=surface),
   eventFX = dDecay(neighs=1, c=1, d0=0.24),
   area=island
   )
\end{verbatim}

Whether the influence of a location that has never had a bird is different from one where a bird has left needs to be decided
according to the application. I'm not sure what the code should do by default. Clearly with nest building, new birds may not 
try to use abandoned nests until the nests are clearly abandoned, or they may re-use empty nests preferentially. I don't know 
enough about birds to answer this. Any of these possibilities could be implemented in code.


\section*{Extending the Framework}

\subsection*{Specifying a New Distance-Decay in the FMD model}

The Euclidean distance-decay effect in the FMD model is a specific case of a transmission function. The parameters of
such a function in general are the two farms of interest and the parameters of the decay function:

\begin{verbatim}
dDecay = function(params,i,j){...}
\end{verbatim}

Now this function needs to get the specific information about the farms in order to compute the transmission factor. One 
technique for this would be for the function to get the data from the global environment, but that ties the function
to a particular named object. By creating this function within another function it can get the data from objects
local to its own environment:

\begin{verbatim}
fCowSheep <- function(data){
  x=data$x
  y=data$y
  foo = function(phirho,i,j){
    phi=phirho[1]
    rho=phirho[2]
    kappa=0.5
    u2=(x[i]-x[j])^2+(y[i]-y[j])^2
    ep2 = exp(phi)^2
    return( exp(-(u2/ep2)^(kappa/2) ) + exp(rho) )
  }
  return(foo)
}
\end{verbatim}

Then to compute the value of the transmission function between farms 3
and 27 in our data we first construct the function using our data and
then call it with the parameters and the farm indexes:

\begin{verbatim}
dDecay = fCowSheep(data)
dDecay(c(0.1,-2.3), 3, 27)
\end{verbatim}

At this point we can even remove {\tt data} from our R session and the function will still work - using the 
values of {\tt x} and {\tt y} stored in its own environment. You can inspect this environment using the {\tt with} function:

\begin{verbatim}
> with(environment(dDecay),ls())
[1] "data" "foo"  "x"    "y"   
\end{verbatim}

In order to use these functions within the framework it is necessary to add some metadata to the function
by creating it with some attributes for the number of parameters and their names:

\begin{verbatim}
fCowSheep <- function(data){
  [...]
  foo = function(phirho,i,j){
    [...]
  }
  attr(foo,"nparam")=2
  attr(foo,"params")=c("phi","rho")
  return(foo)
}
\end{verbatim}

This ensures that the framework takes the right number of parameters from the overall vector of parameters
and feeds it to this function in the right places.

It is also a requirement that the function works correctly for vector values of either \verb|i| or \verb|j|.


\subsection*{Specifying a New Distance-effect For A Space-Time Point Pattern}

The general distance effect in the space-time point pattern case is a function of location and time. In the bird
nesting case the function boiled down to the distance to the nearest present bird at that time. A more complex 
model might be needed for earthquake pattern modelling. This is how we would include a temporal-decay component
to the distance-effect in order to model aftershocks around an initial quake.


$$
h(s,t)=1 + (\theta_s \exp\{-s/\phi_s\}) . (\theta_t \exp\{-t/\phi_t\})
$$


\section*{How Flexible Do We Make This?}


The main fitting function works by constructing the likelihood function and calling optim. Quite simply,
it looks like this:

\begin{verbatim}
fitST = function(start, [other args]){
   L = llikST([other args])
   optim(start,L)
}
\end{verbatim}

The likelihood function works by constructing a lambda function and then looping over it, like this:

\begin{verbatim}
llikST = function([args]){
 lambda = buildLambda([args])
 logl=0
 for(i in 1:length(x)){
    logl = logl + log( 
      lambda(x[i],t[i],params) / integrated(lambda,params,t[i],A)
    )
  }
  logl
}
\end{verbatim}

The lambda construction function constructs the covariate and proximity effect functions and returns
a lambda function that computes their product:

\begin{verbatim}
buildLambda = function([args]){
  covF = covarFunction([args])
  proxF = proxFunction([args])
  lambdaF = function(x,t,params){
    covF(x,t,params[])*proxF(x,t,params[])
  }
  lambdaF
}
\end{verbatim}

The partial log-likelihood for the bird nesting application is:

$$
L_p = \sum^n_{i=1} \log \frac{\lambda(x_i,t_i|H_{t_i})}{\int_A \lambda(x,t|H_{t_i}) \mathrm{d}x}
$$

so in the fully general case, someone has to write a function, lambda, which gets plugged into this:

\begin{verbatim}
L = function(params,x,t,lambda,A){
 logl=0
 for(i in 1:length(x)){
  logl = logl + log( 
     lambda(x[i],t[i],params) / integrated(lambda,params,t[i],A)
    )
 }
 logl
}
\end{verbatim}

The application in the paper splits this into spatial covariates (with no time-dependence) and a nest-proximity effect. With the 
cancellation of $\lambda_0(t)$ the relevant function is then:

$$
\lambda(x,t) = \exp(\beta_1 z(x)) g(x,t|H_t,\beta_2)
$$

where $\beta_1$ and $\beta_2$ are two vectors of parameters.

The functions proposed above support any number of log-linear terms in the first part and a choice of implemented models for the second part.

It is probably just as easy to allow for space-time varying covariates:

$$
\lambda(x,t) = \exp(\beta_1 z(x,t)) g(x,t|H_t,\beta_2)
$$

and this would allow for changes in the nature of the surface to be incorporated. The user would supply
a function of location and time. For example, if at point $(123.2,212.0)$ an erosion event at $t=1.0$ turns the
meadow surface to rock, the function would behave as follows:

\begin{verbatim}
> surface(132.2, 212.0, 0.5)
 [1] meadow 
Levels: grass meadow rock
> surface(132.2, 212.0, 2.5)
 [1] rock 
Levels: grass meadow rock
\end{verbatim}

The next question is whether we can allow for non log-linear covariate terms, and whether we can allow for 
arbitrary event proximity functions.


\section*{Variations}

The FMD and Swine Flu models are so similar that they could be modelled by a single function. The variations are:

For the infection time:

\begin{itemize}
\item assumed known infection time
\item infection time parametrically dependent on data
\end{itemize}

For the removal time:

\begin{itemize}
\item given or assumed known removal time 
\item infection duration parametrically dependent on data
\end{itemize}

Removals and infections may also involve:

\begin{itemize}
\item removal of spatial units after infection
\item possible multiple infections per spatial unit
\end{itemize}

The spatial basis can be:

\begin{itemize}
\item cases are from a finite set of discrete units
\item cases are individuals from a population at a set of locations
\end{itemize}

Covariate options include:

\begin{itemize}
\item selection of covariates and covariate formula for $A_i$ and $B_j$
\item selection of transmission matrix and functional form for $C_{ij}$
\end{itemize}

With Swine Flu each case is related in a many-to-one fashion to the spatial units (regions), but for
the FMD the cases are a one-to-one mapping of cases to spatial units (farms). With the former its necessary to give
two data frames - one of cases and one of regions - but with the latter it can all go in one data frame.

The FMD example could be considered as cases in spatial units with a population of one, with removal. The Swine Flu example
was spatial units with a large population (and no removal).

\end{document}